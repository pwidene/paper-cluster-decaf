\section{Introduction}
\label{s:intro}

As extreme computing architectures continue to evolve, it is 
clear that I/O is an increasingly significant problem.  Projections suggest no
significant parallel file system I/O bandwidth growth through the next 100x
increase in compute capabilities.  The traditional HPC approach of writing
complete checkpoints or analysis dumps for later processing to gain scientific
insights is becoming infeasible.  Even with technologies like Burst Buffers,
limited capacity reduces the usefulness. As such, in situ and in transit
analysis and visualization toolkits with the capability to reduce, process, and
otherwise mitigate the raw increase in data volume and throughput requirements
are increasingly important.

% here we go directly from in situ argument
% to examples of workflows. needs transition. ok.
% also, confusing to go into detail about the existing
% WMSs here and then have a related work section.
% reduce.

Scientific workflows, which 
capture a series of such analytical steps for a
particular scientific discovery goal, usually
by operating on large quantities of data,
have also followed this trend of moving the analytical components
online (ie. executing simultaneously with the driving scientific code).
A number of scientific workflow engines exist.
However, while workflow engines such as
Kepler~\cite{bertram:2006:kepler} and
DAGMan~\cite{Malewicz:2006:dagman}
offer flexible ways to assemble components
with rich functionality to manage the control flow, they have not been
universally adopted due to some of their limitations and complexity in the
HPC environment.

An example illustrating the complexities comes from the Oak Ridge
Leadership Computing Facility (OLCF).  Kepler was used for several workflows
for the fusion simulation users.  While the initial goal was that an internal,
expert resource would create the workflow,
including the data transformation code required
to allow downstream components to operate on the output of
uptream ones (ie. the ``glue code''),
the workflow should be able to be maintained
easily by the application scientists. Unfortunately,
the expert was required regularly as the configuration evolved and scaled.
The complexities of maintaining the components and the glue code as the output
shifted and managing the deployment was too high.
Even then, the workflow operated offline.

One major limitation to existing workflow engines is that
given only a simluation code, a scientist cannot easily
assemble a workflow ``out of the box.''
Indeed, even with the re-use of code that performs
common analytical procedures and with the automation of control flow
allowed by existing engines,
for a scientist to assemble
and launch a workflow using existing tools,
significant effort is required to (a) allow the existing
analytics code
to operate on the specific data formats used in this
particular workflow,
(b) reconcile the discrepancy between the potentially
differing levels of parallelism
of the various components,
and (c) map the workflow components
to the available resources, which in the HPC space
is usually a supercomputer.

% I see a, b, and c as the main problems we are solving with SG
% -> (a) with the any-dimension genericity of tools like select and dim-reduce
% -> (b) with the use of a simple bash script to launch a full workflow, which is
% a common scenario
% -> (c) by the automatic partitioning of data that the components do,
%        without special knowledge required from the scientist.


%not sure where this fits, but not here
\if 0
Distributed-Area Computing
(DAC) workflow management
systems~\cite{tejedor:2015:pycompss,deelman:2015:pegasus} and HPC workflow
management systems~\cite{dorier:2015:in-situ-lessons} may need to interoperate
for some scenarios~\cite{deelman:2015:workflows-report}. While both exist, to
date they have been developed separately in different communities. We have only
just recently begun to develop a plan and a prototype for integrating such
heterogeneous workflow systems. This paper focuses on managing the HPC portion of
these heterogeneous workflows.
\fi

%Figure 1 is an example of such an integration.  It shows a top level
%DAC workflow with one of the nodes in the DAC workflow graph being an HPC
%workflow graph running on a supercomputing center. The interfaces between the
%two WMSs occur from DAC to HPC and from HPC to DAC. The former interface
%requires the exchange of the HPC subworkflow graph, which presumably was
%defined by the user through the interaction with the top-level DAC WMS, and the
%exchange of data products produced by the DAC that need to be further processed
%by the HPC subworkflow. The latter interface, from HPC to DAC, involves the
%transfer of data products only.

% we already talked about the trend of moving workflows online
\if 0
Falling back to Python
scripts managed by the application scientist proved easier and faster to
maintain. While this approach using the parallel file system to stage
intermediate data was sufficient, it is quickly becoming infeasible. The IO
overhead for using the parallel file system is exceeding acceptable runtime
percentages forcing a reduction in output and making scientific insights more
difficult to discover. 
\fi

Proposed infrastructure for online workflows is less mature but can offer
the necessary functionality for assembling workflows
ad-hoc~\cite{zheng:2010:predata,docan:2010:dataspaces,wolf:2002:smartpointer,abbasi:2009:datastager}.
For example, in PreDatA~\cite{zheng:2010:predata}, separate applications are
run to offer different resilience domains, but the actual placement of
operators may be different depending on the performance characteristics requiring
coding changes for each placement decision.
DataSpaces~\cite{docan:2010:dataspaces} offers a way to map data from one
application to another, but it has no mechanisms to manage what is a complete
data set, when it begins in an output stream, and when it ends. The user has to
determine if what is being requested is complete and correct or not.
Data Staging~\cite{nisar:2008:staging} and in particular
data staging with processing
capabilities~\cite{abbasi:2009:datastager,ober:seismic}, are a solid step
forward in providing online functionality for workflows, but do not 
offer a generic approach for their assembly.
SmartPointer~\cite{wolf:2002:smartpointer} and
DataStager~\cite{abbasi:2009:datastager} both offer examples of integrated
workflows, but the construction is completely custom and not portable to other
workflows.

%\ada{this is a very vague claim, doesn't help explain the key
%  problem. the previous paragraph above is, is there more evidence?}

This lack of a robust infrastructure and the difficulty in
using these tools direcly to assemble an
HPC workflow out of the box
based on an existing scientific code,
simulation or other,
and launching it in a way that is
as simple as a single job submission
have limited their adoption.

Newer work is extending these
systems to provide more generic functionality
in workflow assembly and
management~\cite{dayal:2014:flexpath,dayal:2015:escience,lofstead:2012:txn,lofstead:2013:txn-pdsw,lofstead:2012:txn-metadata,lofstead:2014:txn,lofstead:2016:superglue}.
In particular, FlexPath~\cite{dayal:2014:flexpath}
is a publish/subscribe system that allows
two concurrently running MPI executables to
exchange data that is globally partitioned among
their ranks, even when the number of writers and readers differ.
It is implemented as a transport method for the
Adaptable I/O System (ADIOS)~\cite{lofstead:2009:adaptable},
which provides a unified interface to parallel I/O over a rich
variety of underlying data encoding, transport, and placement technologies.
With ADIOS, FlexPath provides a simple interface for MxN communication
between simulation and/or analytical components.
Through ADIOS and FlexPath, data exchange makes use
of the abstraction of a stream, which allows
for asynchronous, MxN I/O between workflow components.

In this work, we leverage the stream-based, MxN
I/O interface offered by ADIOS and FlexPath
to present SuperGlue, a collection of generic workflow components
that allow for the easy assembly and deployment
of a variety of scientific workflows.

SuperGlue allows workflows to be assembled
by using existing, generic
data transformation and 
analysis components
which to not require any code changes
for specific workflows;
the only code changes required when
using SuperGlue are 
for redirecting of the output of interest
in the driving simulation code through ADIOS.

SuperGlue components solves a number of the aforementioned problems in
current online workflow-oriented technologies.
First, they leverage the self-describing
property of data exchanged through ADIOS to
allow for operation on data having very different formats.
In scientific workflows, the data of interest
is generally composed of large, multi-dimensional
arrays; in SuperGlue, this translates to
the ability to operate, as much as possible,
on data having any number of dimensions.
To address the difficulty and complexity
in mapping workflows to supercomputers using
existing workflow technology, which is targeted
at a variety of resources outside of HPC, such as Grid and Web
resources,
SuperGlue reduces the
launch of an entire workflow to the submission of a single job script.
Finally, it uses the MxN publish/subscribe
ability of FlexPath,
as well as consistent semantics describing
the data that are passed through the components,
to exchange data while preserving semantic integrity
even in the presence of varying levels of parallelism 
at different stages of the workflow.

Rather than providing a complete collection of such components,
we provide here the basic building
blocks of what we hope will be a library of components that can
eventually be used to assemble
a rich variety of workflows, driven by simulation codes
exporting a variety of different data formats.

\if 0
\ada{sell the work. highlight something that says: your contributions are the design (you thought
  about what's the right lower level functionality you can leverage to
  make this work), collection of generic SG filters, and demonstration
through large-scale deployment of three different workloads that the
approach is 
expressiveness, can impact productivity since components can be reused
with minimal intervention, and doesn't impeed performance and
experimentation at scale. }
\fi
To demonstrate the functionality of the SuperGlue approach,
we deploy three SuperGlue workflows based on three different
scientific simulations, namely
LAMMPS~\cite{plimpton:1997:lammps},
GTCP~\cite{lin:gtc},
and GROMACS~\cite{hess2008gromacs}.
We show that SuperGlue allows workflows to scale well and that
simple experiments allow for the selection of
appropriate levels of parallelism at different
stages of the workflow.

One disadvantage that might be perceived in the fine-grain componentization
of a workflow that naturally comes with the use of SuperGlue
is the potential loss in performance.
However, we show through a performance comparison of
a componentized, generic workflow with a corresponding
ad-hoc workflow which uses a monolithic analytical component
that the componentization of analytical routines
does not lead to a noticeable change in the
overall performance of the workflow.

%While the online shows a lot of potential to improve the scientific discovery
%process, it requires reworking existing workflows. Since this is a non-trivial
%task, tools that can simplify creating an efficient workflow within a single job
%submission is crucial to adoption.
%\ada{text before not clear}
%New workflows can
%be made from scratch online lowering the barrier, but there are still costs
%associated with using existing components that may or may not integrate well
%into the online structure.

%The ``killer app'' for workflows, both offline and online, is being able to
%submit a single job that is capable of running the entire workflow. Existing
%techniques to make this work requires machine reservations or special
%permission for automated connections between the management machine and the
%large scale platform.  SuperGlue's approach bridges the connectivity problems
%and addresses the data type mismatches common when data flows from one
%component to the next.

%========================

\if 0
To address the performance mismatch, Integrated Application Workflows (IAWs)
are being developed. The easiest way to think of these IAWs is the Unix/Linux
shell pipe operator to connect commands. The shell connects stdout of one
program to stdin of the next with the assumption that each component in the
chain can operate in this mode. For tasks at this scale, this approach works
well. For the scientific workflows we are targeting, we have processes spread
across potentially 10,000s of nodes connected to other components also running
on multiple nodes. Unlike the command-line tools, none of these components
generally have the ability to shift their input or output from how it is
written to connect to a new online component without potentially significant
code changes. The challenge with these workflows is dealing with the lack of a
persistent store to stage intermediate data, interfaces for communicating data
state and availability, and data organization changes required before a
component can process data.  Each of these has been investigated by different
projects over the last several years.

Several frameworks have been developed that offer some functionality for
supporting online workflows. The more advanced examples incorporate some data
processing components, sometimes of limited scope, for performing in situ or in
transit processing. The ADIOS IO library~\cite{lofstead:2009:adaptable} was designed with
this use case in mind. The HDF5 Virtual Object Layer
(VOL)~\cite{chaarawi:2013:hdf5-vol} was developed to support similar
functionality.

Several efforts to work through some of the issues related to IAWs have been
investigated~\cite{karimabadi:2013:catalyst,whitlock:2011:libsim,Glean,dayal:2014:flexpath,dreher:2016:bredala,zheng:2010:predata},
and much on-going research in the space promises to expand on some of the
needed techniques for management and placement.  Some
scientific codes have been addressing similar constraints for years by
in-lining analytics functions and performing complicated MPI communicator
subdivisions to allow simulation and analytics to co-exist.  One key
observation, however, is that there is a lack of portability to the resulting
implementations; they require a great deal of tuning and/or runtime placement
control to make them function as desired.
\fi


\if 0
Unlike existing components used in IAWs, SuperGlue
components do not have a fixed data type. \ada{if this is the key
  difference, then somewhere mention the fixed type limitation. }
 This one change enables using these
components on completely different kinds of simulations that share nothing in
their output format. Key to making this work is using a typed transport
mechanism between different components. Many options exist for these transports
and the particular mechanism selected is not critical. 

\ada{summarize some insight here: what do you leverage/need in order
  to be able to templetize the glue components, and then to permit the
same approach to work for instantiating analytics... typed transport
is one, anything else?}
In an effort to better understand when choosing one approach for constructing a
workflow is perferable to another, we have developed a collection of ``filters''
we call SuperGlue that allow us to experiment with the tradeoffs. This paper
describes our components and examines some different workflows constructed using
different techniques to compare when the costs and benefits for one approach are
a better tradeoff than for others. 
\ada{sell the work. highlight something that says: your contributions are the design (you thought
  about what's the right lower level functionality you can leverage to
  make this work), collection of generic SG filters, and demonstration
through large-scale deployment of three different workloads that the
approach is 
expressiveness, can impact productivity since components can be reused
with minimal intervention, and doesn't impeed performance and
experimentation at scale. }
Our main contribution is the evaluation of
these workflows constructed using different engines and techniques to understand
when different approaches become viable, such as IAWs, or if cobbling something
together is sufficient or even superior for certain metrics.
%analysis and manipulation tools that can be chained together to form a variety
%of real-time workflows providing analytical results during the execution of the
%primary scientific code.

While the main technical development contribution of this work is the design of
generic {\em glue code}, the {\em analytical} components we use in the
workflows we present here are designed using the same approach. For example, an
index selection tool is more of a glue component while a histogram component is
more analytical.  Still, we design both types of components as single MPI
executables with similar input and output interfaces launched using similar
configuration options.  By blurring the line between generic glue components
and generic analytical components, we present a flexible way to assemble
scientific workflows without having to write any additional code. 
\fi

The rest of the paper is organized as follows. First is a survey or related
work in~\autoref{s:related}. Second, in~\autoref{s:workflow}, is a
discussion of the three workflows we use to drive our insights. Next is an
overview of the concepts behind SuperGlue in~\autoref{s:design}.
In~\autoref{s:reusable-components}, we provide an overview of the reusable
components created. In~\autoref{s:integrating-reusable-components}, we
discuss the challenges of integrating reusable components.  An evaluation is
presented next in~\autoref{s:eval}. Finally is a discussion of Conclusions
and Future Work in~\autoref{s:conclusion}.

