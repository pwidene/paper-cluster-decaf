\section{Integrating Reusable Components}
\label{s:integrating-reusable-components}

To allow the simulation codes used to work
with SuperGlue, we had to modify their output
routines to use ADIOS.
ADIOS expects multi-dimensional data
to be packed as a linear array.
LAMMPS and GTCP do this packing natively,
but we had to add it to GROMACS.
Roughly 70 lines of code are required
to allow each simulation to work with SuperGlue.
The code changes made to GROMACS version 4.6.7
can be seen in the public SuperGlue
repository~\cite{champsaur:superglue-repo}.

In general, in order to work with SuperGlue components, 
simulations have to specify through ADIOS the logical
dimensions of their data and optionally create
corresponding headers to label them and their indices.

\subsection{Demonstrating in the Workflows}

We built the LAMMPS workflow, illustrated
in~\autoref{fig:lammps-workflow},
using only LAMMPS and SuperGlue components.
We annotate the figure with details about how
the data is manipulated at each step.

Data arrives from LAMMPS at the first SuperGlue component, Select, which
extracts the velocity components from the raw output of the simulation. From
Select, data is sent to Magnitude, which computes
the magnitudes of the velocities. Magnitude
outputs one-dimensional data, an array of the magnitudes it calculates, to the
final component, Histogram, which expects one-dimensional data as input. The
end result of this workflow is a series of histograms of the total velocities
of the particles. There is one histogram created at each timestep at which the
simulation would normally dump its data to disk.


The GTCP workflow is illustrated in~\autoref{fig:gtcp-workflow}, and it too was assembled only from the
simulation and SuperGlue components. Note that the workflows primarily differ
in the data formats output by the simulations.

From GTCP, data first arrives at an instance of Select, which extracts
one quantity of interest out of the 7 properties that describe each gridpoint.
This quantity was arbitrarily chosen as the ``perpendicular
pressure,'' or pressure of the plasma perpendicular to the flow in the grid
point of interest. Even if it contains only perpendicular pressures, the output
of Select is still three-dimensional since this component maintains the
original dimensions of its input. Because the Histogram component expects
one-dimensional input, we first send the output of Select through two instances
of our Dim-Reduce component, each of which eliminates a single dimension of the
array without changing its total size. The final component, Histogram, outputs
a histogram of the perpendicular pressures of all grid points at each timestep
at which the simulation would normally output its data to disk.

