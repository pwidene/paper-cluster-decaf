\section{Integrating Reusable Components}
\label{s:integrating-reusable-components}

In order to use some of the same components in both workflows, we had to
slightly modify the output stages of the scientific codes driving them. Because
in both workflows, the first component to receive the simulation data is
Select, each simulation has to write a header of its quantities in the
dimension to be selected from. Also, normally LAMMPS packs its two-dimensional
output into a single array. We eliminated this packing to reduce the downstream
decoding complexity by leaving the actual data structure rather than an encoded
form.
Both simulations had to be
modified to use ADIOS for output. While the ADIOS integration was
not difficult because the ADIOS interface is simple, it was the most
significant change made to the
simulations.

In general, in order to work with SuperGlue components, 
simulations have to specify through ADIOS the logical dimensions of their data
and optionally create corresponding headers to label them and their indices.

\subsection{Demonstrating in the Workflows}

We built the LAMMPS workflow, illustrated in~\autoref{fig:lammps-workflow},
using only LAMMPS and SuperGlue components.
We annotate the figure with details about how
the data is manipulated at each step.

Data arrives from LAMMPS at the first SuperGlue component, Select, which
extracts the velocity components from the raw output of the simulation. From
Select, data is sent to Magnitude, which computes
the magnitudes of the velocities. Magnitude
outputs one-dimensional data, an array of the magnitudes it calculates, to the
final component, Histogram, which expects one-dimensional data as input. The
end result of this workflow is a series of histograms of the total velocities
of the particles. There is one histogram created at each timestep at which the
simulation would normally dump its data to disk.


The GTCP workflow is illustrated in~\autoref{fig:gtcp-workflow}, and it too was assembled only from the
simulation and SuperGlue components. Note that the workflows primarily differ
in the data formats output by the simulations.

From GTCP, data first arrives at an instance of Select, which extracts
one quantity of interest out of the 7 properties that describe each gridpoint.
This quantity was arbitrarily chosen as the ``perpendicular
pressure,'' or pressure of the plasma perpendicular to the flow in the grid
point of interest. Even if it contains only perpendicular pressures, the output
of Select is still three-dimensional since this component maintains the
original dimensions of its input. Because the Histogram component expects
one-dimensional input, we first send the output of Select through two instances
of our Dim-Reduce component, each of which eliminates a single dimension of the
array without changing its total size. The final component, Histogram, outputs
a histogram of the perpendicular pressures of all grid points at each timestep
at which the simulation would normally output its data to disk.

