Scientific simulation workflows have traditionally used
the storage array as a staging place between the components
that make up the workflows and employed a high-level
task manager to trigger these components.
Recent work has been investigating moving these
workflows to purely online models, thus avoiding the
well-known performance bottleneck of storage arrays.
While existing workflow management systems
can successfully organize and schedule such
online workflows, the components are generally
either written from scratch, or they at least require
significant changes in order to fit into different workflows
in which the components operate on different
data formats.

In this work, we present Superglue, a set of generic workflow
components that can be used to assemble a variety of online,
static (ie. submitted as single jobs) workflows from start to finish without
the need for re-compilation. We leverage the self-describing,
process size- and placement-independent properties of data exchanged
by the Adaptable I/O System (ADIOS) to assemble and run three
Superglue workflows based on three different and widely-used simulation
codes exporting different data formats.
We show that the performance cost of a componentized
approach to assembling workflows is small, especially when considering
the advantage that is the ability to assemble complete workflows
``out of the box.''
While keeping in mind performance considerations, we present the
building blocks as well as an approach to expand a library of such components,
which we hope can eventually meet a wide variety of analytical needs
in the scientific computing community, in the common scenario of
submitting a workflow as a single job to a supercomputer scheduler.


\if{0}
The workflows are assembled using
some of the same components, but leading to very different analytical results.
In this way, we present the building blocks of a library of generic components


Scientific simulation workflows are becoming a 
Considering the well-known bottleneck that is the persistent storage array,


Scientific simulation workflows have traditionally focused on using the storage
array as a staging place between components and employ a high level task
manaager to trigger components. Recent work has been investigating moving these
workflows to purely online models avoiding the storage array bottlenecks. This
transition is not cost free, but there are also potentially enormous advantages.
For example, by moving online, more the data movement is limited by bisection
bandwidth rather than storage bandwidth. However, making existing tools work
online may take some modification and adding in the ``glue'' components required
to translate the output from one component to the input for the next are no
longer simple scripts.

This paper examines these tradeoffs and lays out when the advantages may make
moving purely online worthwhile even though not being forced by platform
limitations. Support tools necessary to support both models are also discussed
including offering standardized ``glue'' components for online workflows making
the transition easier.
\endif
