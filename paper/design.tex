\section{Design}
\label{s:design}

In this project, our main
goal is obtain noticeable scaling of
\fakeroot for increasing
core count on a single node.
Because \make works on an entire code
base at a time, we will focus on
\textit{strong}, rather than \textit{weak} scaling.
Indeed, testing for weak scaling would involve
testing the performance of \make with code
bases of increasing sizes, trying to keep the amount
of work per core the same; though theoretically possible,
this is a much larger and more complex project,
unreasonable in the scope of this class.
If time permits, then we will
move on to utilizing multiple
nodes in a distributed fashion.
For a single node, these are the
following design decisions we
have come up with:
\squishlist
\item We will rely on the idea
of process pool (per-cpu threads)
with many-to-many messaging model to handle
multiple file-manipulation operations
in parallel.
\item Since \fakeroot relies on
a single gigantic hash table, we
will try to use a concurrent
hash table with NUMA awareness,
so that multiple server threads
can update the table in parallel.
If possible, we will try to use
lock-free hash tables.
\item In the multiple-node case,
we can leverage the idea of
map-reduce model~\cite{mapreduce}
with distributed shared memory~\cite{grappa}.
\squishend
