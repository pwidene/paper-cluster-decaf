\section{Conclusions and Future Work}
\label{s:conclusion}

This paper presents SuperGlue, a demonstration of making generic, reusable
components for scientific simulations. By decomposing the operations into small
chunks, we can achieve components that can be reused, without modification, for
a variety of different workflows. In this work we investigate using a
stream-based structure with generic components to achieve easier to build and
use workflows.  Stream-based, generic workflow components should be designed so
as to allow for the greatest variety in their arrangement and for a maximum
number of downstream subscribers. Designing components with the ability to
handle data having any number of dimensions provides a very useful way to link
them together. Maintaining a high level of semantics upstream, for example by
labeling dimensions and certain quantities inside of these dimensions, gives a
good understanding of the data to downstream components. There is a need for
components that organize the data in a format that downstream components can
understand.

Through the demonstration of generating a velocity histogram for LAMMPS,
a pressure histogram for GTCP,
and a distribution of the spread of the atoms for a GROMACS
experiment,
we demonstrate reusing the same components
over different data formats and application types.

While this work leverages ADIOS and the FlexPath transport, this is not the
only approach for addressing reusable components. Other, similar approaches can
also work well. However, in this case, the data annotation provided by this
connection infrastructure helps enable reusable components by offering the
necessary metadata to perform the general operations.

Future work on this project involves
expanding the library of generic components
to include a variety of other analytical operations.
In particular, the SuperGlue components presented
in this paper
result in an output dataset having either the same
size or a smaller size as the input.
Analytical procedures that lead
to an increase in data size, such as
all-pairs calculations, are common and
can be implemented using the SuperGlue approach.

Also common are operations that require the in-memory
buffering of a dataset throughout a series
of timesteps. An example of this is the
calculation of root-mean-square
deviations of atomic positions from a fixed,
starting set of positions.

Two key limitations of SuperGlue are
that at the moment,
all components are launched at once, and
that workflows are not able to branch.

To turn SuperGlue into a true Workflow Management
System, we hope to leverage
ADIOS' ability to have several ``write groups''
so as to allow for the development
of a {em Fork} component
that would permit
the creation of workflows
described by directed acyclic graphs.
And, to manage the execution of workflows
over longer periods of time,
we plan on investigating
the incorporation of SuperGlue into
higher-level workflow management systems
such as Kepler and DAGMan.

The components we have developed in this research cover only a small portion of
the procedures that computational scientists need for their
workflows. Eventually, we wish to allow for the development of a large
collection of generic workflow components. We can take steps in this direction
by building on our existing components. For example, {\em Magnitude} performs a
relatively simple operation on multi-dimensional data, where one dimension
spans a number of quantities involved in each instance of the operation. This
model can fit any number of operations involving a repeating, fixed number of
quantities, and it can even be made to work with a formula specified by the
user at runtime. This opens the door to a large family of generic components.

\if 0
Finally, while we have kept performance in mind in the development of these
components, performance optimization is not yet the focus of this research. In
the design of any generic tool however, the question of performance inevitably
arises. Indeed, designing tools that are not meant to operate on a specific
format of input data can easily impact performance. For example, {\em
Dim-Reduce} performs the same amount of computation whether it re-arranges data
or not. In the long run, optimizing these components will involve detecting
such situations where they can avoid performing unnecessary iterations and data
manipulation.
\fi
