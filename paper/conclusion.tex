\section{Conclusions and Future Work}
\label{s:conclusion}

This paper presents SuperGlue, a demonstration of making generic, reusable
components for scientific simulations. By decomposing the operations into small
chunks, we can achieve components that can be reused, without modification, for
a variety of different workflows. In this work we investigate using a
stream-based structure with generic components to achieve easier to build and
use workflows.  Stream-based, generic workflow components should be designed so
as to allow for the greatest variety in their arrangement and for a maximum
number of downstream subscribers. Designing components with the ability to
handle data having any number of dimensions provides a very useful way to link
them together. Maintaining a high level of semantics upstream, for example by
labeling dimensions and certain quantities inside of these dimensions, gives a
good understanding of the data to downstream components. There is a need for
components that organize the data in a format that downstream components can
understand. And, designing specific disk writer components removes the need to
temporarily modify analytics components to let them also act as disk writers.

Through the demonstration of generating a velocity histogram for LAMMPS, the
molecular dynamics simulation, and a pressure histogram for GTC, the particle
in cell fusion reactor simulator, we achieved reusing the same components in
very different data formats and application types.

While this work leverages ADIOS and the FlexPath transport, this is not the
only approach for addressing reusable components. Other, similar approaches can
also work well. However, in this case, the data annotation provided by this
connection infrastructure helps enable reusable components by offering the
necessary metadata to perform the general operations.

There are a number of improvements we can make to our current implementation to
have at our disposal more robust and flexible workflow components. As mentioned
earlier, reading and writing dimension labels at each step in the workflow
provides more information to downstream components and presents a clear
advantage. The ADIOS interface includes the ability to send output to multiple
destinations by having several ``write groups.'' We wish to explore the
possibility of a {\em Fork Component} that would use this functionality of
ADIOS to allow the creation of branched workflows.

The components we have developed in this research cover only a small portion of
the procedures that computational scientists need for their
workflows. Eventually, we wish to allow for the development of a large
collection of generic workflow components. We can take steps in this direction
by building on our existing components. For example, {\em Magnitude} performs a
relatively simple operation on multi-dimensional data, where one dimension
spans a number of quantities involved in each instance of the operation. This
model can fit any number of operations involving a repeating, fixed number of
quantities, and it can even be made to work with a formula specified by the
user at runtime. This opens the door to a large family of generic components.

Finally, while we have kept performance in mind in the development of these
components, performance optimization is not yet the focus of this research. In
the design of any generic tool however, the question of performance inevitably
arises. Indeed, designing tools that are not meant to operate on a specific
format of input data can easily impact performance. For example, {\em
Dim-Reduce} performs the same amount of computation whether it re-arranges data
or not. In the long run, optimizing these components will involve detecting
such situations where they can avoid performing unnecessary iterations and data
manipulation.

