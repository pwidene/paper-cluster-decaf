\renewcommand{\ttdefault}{pxtt}

\newcommand{\URL}{\url}
\newcommand{\cc}[1]{\mbox{\smaller[0.5]\texttt{#1}}}

%\clubpenalty=10000
%\widowpenalty=10000

%\linespread{1.2}

\fvset{fontsize=\scriptsize,xleftmargin=8pt,numbers=left,numbersep=5pt}

\input{code/fmt}
\newcommand{\figrule}{\hrule width \hsize height .33pt}
\newcommand{\coderule}{\vspace{-0.4em}\figrule}

\setlength{\abovedisplayskip}{0pt}
\setlength{\abovedisplayshortskip}{0pt}
\setlength{\belowdisplayskip}{0pt}
\setlength{\belowdisplayshortskip}{0pt}
\setlength{\jot}{0pt}

\def\Snospace~{\S{}}
\renewcommand*\sectionautorefname{\Snospace}
\def\sectionautorefname{\Snospace}
\def\subsectionautorefname{\Snospace}
\def\subsubsectionautorefname{\Snospace}
\def\chapterautorefname{\Snospace}
%\renewcommand{\figurename}{Fig.}
%\def\figureautorefname{\figurename}
\newcommand{\subfigureautorefname}{\figureautorefname}

%\numberwithin{equation}{section}
\newcommand{\yes}{Y}
\newcommand{\no}{}

% sema
\newcommand{\shl}{\ \cc{<}\cc{<}\ }
\newcommand{\shr}{\ \cc{>}\cc{>}\ }

\if 0
\renewcommand{\topfraction}{0.9}
\renewcommand{\dbltopfraction}{0.9}
\renewcommand{\bottomfraction}{0.8}
\renewcommand{\textfraction}{0.05}
\renewcommand{\floatpagefraction}{0.9}
\renewcommand{\dblfloatpagefraction}{0.9}
\setcounter{topnumber}{10}
\setcounter{bottomnumber}{10}
\setcounter{totalnumber}{10}
\setcounter{dbltopnumber}{10}
\fi

\newif\ifdraft\drafttrue
\newif\ifnotes\notestrue
\ifdraft\else\notesfalse\fi

% ref. http://en.wikibooks.org/wiki/LaTeX/Colors
  \newcommand{\TK}[1]{\textcolor{LimeGreen}{TK: #1}}
  \newcommand{\SK}[1]{\textcolor{Green}{SK: #1}}
  \newcommand{\AC}[1]{\textcolor{Brown}{AC: #1}}
  \newcommand{\CM}[1]{\textcolor{Orange}{CM: #1}}
  \newcommand{\XXX}[1]{\textcolor{red}{XXX: #1}}
  \newcommand{\TODO}[1]{\textcolor{Melon}{TODO: #1}}
  \newcommand{\TODONEXT}[1]{}

%% Ensure ligatures (e.g., ``fine official flag'') can be copy/pasted from PDF.
\input{glyphtounicode}
\pdfgentounicode=1

\newcolumntype{R}[1]{>{\raggedleft\let\newline\\\arraybackslash\hspace{0pt}}p{#1}}

% include macros
\newcommand{\includepdf}[1]{
  \includegraphics[width=\columnwidth]{#1}
}
\newcommand{\includeplot}[1]{
  \resizebox{\columnwidth}{!}{\input{#1}}
}

% list
\newcommand{\squishlist}{
\begin{itemize}[noitemsep,nolistsep]
  \setlength{\itemsep}{-0pt}
}
\newcommand{\squishend}{
  \end{itemize}
}

%%
%% NOTE.
%%  to use circled number in caption, use
%%   (e.g., \protect\C{1})
%%
\usepackage{tikz}
\newcommand*\C[1]{%
\begin{tikzpicture}[baseline=(C.base)]
\node[draw,circle,inner sep=0.2pt](C) {#1};
\end{tikzpicture}}

\newcommand*\BC[1]{%
\begin{tikzpicture}[baseline=(C.base)]
\node[draw,circle,fill=black,inner sep=0.2pt](C) {\textcolor{white}{#1}};
\end{tikzpicture}
}

\usepackage{xstring}
\newcommand{\PP}[1]{
\vspace{2px}
\noindent{\bf \IfEndWith{#1}{.}{#1}{#1.}}
}

\newcommand{\PS}[1]{
\vspace{2px}
\noindent{\bf #1}
}

%%
%% Paper specific commands
%%
\newcommand{\densetbl}{\addtolength{\tabcolsep}{-3pt}}
\newcommand{\densetblend}{\addtolength{\tabcolsep}{3pt}}

\newcommand{\fx}[1]{\mbox{\smaller[0]\textsc{#1}}}
\newcommand{\rot}[1]{\rotatebox[origin=c]{90}{#1}}

\newcommand{\V}{\checkmark}
\newcommand{\N}{-}
\newcommand{\x}{$\times$\xspace}


\newcommand{\mem}{\mbox{\cc{RAMDISK}}\xspace}
\newcommand{\fakeroot}{\mbox{\cc{FakeRoot}}\xspace}
\newcommand{\make}{\mbox{\cc{make}}\xspace}
\newcommand{\buildbot}{\mbox{\cc{Buildbot}}\xspace}
\newcommand{\faked}{\mbox{\cc{faked}}\xspace}
\newcommand{\libfake}{\mbox{\cc{libFakeroot}}\xspace}
\newcommand{\stat}{\mbox{\cc{stat}}\xspace}
\newcommand{\chmod}{\mbox{\cc{chmod}}\xspace}

% file system
\newcommand{\tmpfs}{\mbox{\cc{tmpfs}}\xspace}

% lock types
\newcommand{\rwsem}{\mbox{\cc{rw\_semaphore}}\xspace}
\newcommand{\seqlock}{\mbox{\cc{seqlock\_t}}\xspace}

\newcommand{\rename}{\mbox{\cc{rename()}}\xspace}
\newcommand{\open}{\mbox{\cc{open()}}\xspace}
\newcommand{\create}{\mbox{\cc{create()}}\xspace}
\newcommand{\unlink}{\mbox{\cc{unlink()}}\xspace}
\newcommand{\readdir}{\mbox{\cc{readdir()}}\xspace}
\newcommand{\iteratedir}{\mbox{\cc{iterate\_dir()}}\xspace}


\newcommand{\ra}[1]{\renewcommand{\arraystretch}{#1}}

\newcommand{\MBpS}{MB/s\xspace}
