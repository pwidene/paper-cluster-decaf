\section{Related Work}
\label{s:related}

Existing workflow systems have typically only been able to offer generic,
reusable components when the workflow system is for a particular niche with a
fixed datatype and standardized interfaces. For example, enterprise document
processing systems may all work against a single database with each user only
seeing their current worklist. As documents are processed, they are moved to
the next work queue, or completed state, according to hand-coded
rules~\cite{mckesson-workflow}. The Workflow Management Coalition~\cite{wfmc}
has developed standards to make enterprise process workflows more portable.
These standards are not intended to make components reusable, but to make
different workflow engines able to inter-operate or to port a workflow from one
engine to another.  The actual communication interfaces and data types are left
to the components themselves.

More directly related to this work and the scientific community are workflow
engines and frameworks custom-made for the parallel computing environment.
Pegasus~\cite{mullender:pegasus} and DAGMan~\cite{Malewicz:2006:dagman} work
together to offer an engine to execute a workflow and a front-end system to
construct the workflow process itself.
However, the focus in DAGMan is on specifying dependencies between
the jobs involved in the workflow, so as to execute
components only when required and provide resilience.
In contrast, in \sys, the entire workflow is executed at once, with
FlexPath allowing readers and writers to block until
the corresponding writer or reader is available for data exchange.
Furthermore, the Pegasus/DAGMan engine does not provide
generic, reusable components.

As an alternative to DAGMan, Swift~\cite{wilde2011swift}
is a scripting language that allows ordinary applications
to be composed into parallel scripts, and eventually into
workflows, with dependencies specified in the script.
However, these applications themselves
must be written outside of the Swift script.

Kepler~\cite{bertram:2006:kepler} offers a nice GUI for assembling different
kinds of scientific workflows. It offers different directors to manage how the
workflow will be executed. Each component is an actor with channels connecting
actors. As mentioned in the Introduction, complex workflows using Kepler have been
assembled for many communities, including fusion science.
However, the large collection of components
that come with Kepler are mainly designed
to work in a single Java Virtual Machine instance;
using HPC-scale components requires significant effort
both in coding the custom components and in
allowing the I/O methods used to be understood by the
higher-level
Kepler engine.

To address much of the complexity of communicating between separate parallel
components, inline approaches are being investigated.
We use ``inline'' in the sense of ``linked into the same process.''
Catalyst~\cite{karimabadi:2013:catalyst} offers a way to integrate the
ParaView~\cite{Moreland:2008:paraview} analysis and visualization system
directly into the simulation executable. Catalyst strips out many features to
reduce the memory footprint and then requires explicit calls from the host
application into Catalyst routines with predictable data types on in-memory
data structures. While this can work for limited kinds of data processing,
it clearly cannot take advantage of additional resources
available for off-site analysis, instead taking cycles
away from the main scientific code.

%above is obvious enough I think
\if 0
two
limitations can cause problems. First, in an internal Sandia project in 2012,
the CTH shock physics code used ParaView both in situ and also in transit. The
in situ integration saw the executable grow from 30 MB to 300 MB and the
scalability was strictly limited due to design flaws in ParaView for in situ
use. While this project prompted the creation of Catalyst, this stripped down
version of ParaView does not address all concerns. There is still a memory
footprint overhead and a runtime pause in the simulation progress for the
analysis and visualization to run.
\fi

Libsim~\cite{whitlock:2011:libsim} has a similar relationship to
VisIt~\cite{Riedel:2007:visit} as Catalyst has to ParaView. Both Libsim and
Catalyst have a strict limitation that offline workflows do not. In both cases,
because they are running on the same nodes as the simulation, time series
analysis and visualization can be difficult or impossible. The potential
scaling impact on the simulation because of limitations in Libsim or Catalyst
can prevent their use for the most important use cases at extreme scale.

A middle path between in situ and offline processing was investigated in
PreDatA~\cite{zheng:2010:predata}.
This work demonstrates that the placement of the analytics can significantly
affect the performance of workflows, and that this placement can be determined
in part by the communication characteristics of the analytics components.

% need re-work.
% removing for now since long already.
\if 0
Glean~\cite{vishwanath:2011:glean}, Nessie~\cite{oldfield:lwfs-data-movement},
and Mercury~\cite{Soumagne:2013:mercury} are intended to facilitate offering
portable workflows across different interconnect technologies. While their
origins may not have been directly for addressing workflows, they have been
re-purposed to address this field. Rather than offering anything related to
managing data types, these tools simply offer a portable way to construct
workflows.
\fi

Companion work to \sys within this same project is
Bredala~\cite{dreher:2016:bredala}. This work presents an attempt to build a
data model for in situ workflows. It has some similarity to
FFS~\cite{eisenhauer:2011:ffs}. Unlike FFS, which is part of a much more
complex infrastructure for typed messaging between distributed processes,
Bredala strictly focuses on a data model that can preserve
semantic integrity across redistributions.

Overall, while all of these efforts are addressing different portions of the
online workflows puzzle, none of them address the idea of general,
reusable data transformation and analysis components for the assembly
of entire HPC workflows out of the box.

