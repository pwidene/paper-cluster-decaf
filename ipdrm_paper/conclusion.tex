\section{Conclusions and Future Work}
\label{s:conclusion}

This paper presents \sys, a demonstration of making generic, reusable
components for scientific workflows. By decomposing the operations into small
chunks, we achieve components that can be reused, without modification, for
a variety of different workflows. In this work, we investigate using a
stream-based pipeline with generic components to achieve easier to build and
use workflows. Designing components having the ability to
handle data having any number of dimensions, 
and to operate on various streams and arrays
passed to them at run time,
provides a very useful way to link
them together. Maintaining a high level of semantics upstream, for example by
labeling dimensions and certain quantities inside of these dimensions, gives a
good understanding of the data to downstream components. There is a need for
components that re-organize the data in a format that downstream components can
understand.

Through the demonstration of generating a velocity histogram for LAMMPS,
a pressure histogram for GTCP,
and a distribution of the spread of the atoms for a GROMACS
experiment,
we demonstrate reusing the same components
over different data formats and application types.

While this work leverages ADIOS and the FlexPath transport, this is not the
only approach for addressing reusable components. Other, similar approaches can also work well. However, in this case, the data annotation provided by this
connection infrastructure helps enable reusable components by offering
necessary metadata to perform general operations.

The approach presented here is limited
to in situ workflows only, which are launched
all at once. While there are many other
commonly used execution models, this
approach is becoming increasingly useful,
considering the increasing significance of
I/O bottlenecks.

Future work involves
expanding the generic components library 
to include a variety of other analytical operations.
In particular, the \sys components presented
in this paper
result in an output dataset having either the same
size or a smaller size as the input.
Analytical procedures that lead
to an increase in data size, such as
all-pairs calculations, are common and
can be implemented using the \sys approach.

To enrich \sys into a true Workflow Management
System, we hope to leverage
ADIOS' ability to have several ``write groups''
so as to allow for the development
of a {\em Fork} component
that would permit
the creation of workflows
described by directed acyclic graphs.
And, to manage the execution of workflows
over longer periods of time,
we plan on investigating
the incorporation of \sys into
higher-level workflow management systems
such as Kepler and DAGMan.

\if 0
Finally, while we have kept performance in mind in the development of these
components, performance optimization is not yet the focus of this research. In
the design of any generic tool however, the question of performance inevitably
arises. Indeed, designing tools that are not meant to operate on a specific
format of input data can easily impact performance. For example, {\em
Dim-Reduce} performs the same amount of computation whether it re-arranges data
or not. In the long run, optimizing these components will involve detecting
such situations where they can avoid performing unnecessary iterations and data
manipulation.
\fi
