\section{Conclusions and Future Work}
\label{s:conclusion}

This paper presents \sys, a demonstration of making generic, reusable
components for scientific workflows. By using a stream-based pipeline and decomposing the operations into small
chunks, we achieve components that can be reused, without modification, for
a variety of different workflows. 
These components 
handle data having any number of dimensions
and operate on various streams and arrays
passed to them at run time.
Maintaining high level semantics upstream, by
labeling dimensions and certain quantities inside of these dimensions, gives
good data understanding to downstream components.
%There is a need for
%components that re-organize the data in a format that downstream components can
%understand.

Through the demonstration of generating a velocity histogram for LAMMPS,
a pressure histogram for GTCP,
and a distribution of the spread of the atoms for a GROMACS
experiment,
we demonstrate reusing the same components
over different data formats and application types.

While this work leverages ADIOS and the FlexPath transport, this is not the
only approach for addressing reusable components. Other, similar approaches can
also work. However, in this case, the data annotation provided by this
infrastructure simplifies creating reusable components.

The components presented here are limited to in situ workflows with all
components running simultaneously. However, introducing new components that write
and read from storage as part of a workflow can break that dependency and are a
superficially simple addition. Future work will investigate these challanges.

Other future work involves
expanding the generic components library 
to include a variety of other analytical operations.
In particular, the \sys components presented
in this paper
result in an output dataset having either the same
size or a smaller size as the input.
Analytical procedures that lead
to an increase in data size, such as
all-pairs calculations, are common and
can be implemented using the \sys approach. Understanding the implications of
data size growing components would further demonstrate the generality of this
approach.

To enrich \sys into a true Workflow Management
System, we hope to leverage
ADIOS' ability to have several ``write groups''
so as to allow for the development
of a {\em Fork} component
that would permit
the creation of much richer workflows
described by directed acyclic graphs,
such as those used to evaluate~\cite{sim2015analyzethis}.
And, to manage the execution of workflows
over longer periods of time,
we plan on investigating
the incorporation of \sys into
higher-level workflow management systems
such as Kepler and DAGMan.

\if 0
Finally, while we have kept performance in mind in the development of these
components, performance optimization is not yet the focus of this research. In
the design of any generic tool however, the question of performance inevitably
arises. Indeed, designing tools that are not meant to operate on a specific
format of input data can easily impact performance. For example, {\em
Dim-Reduce} performs the same amount of computation whether it re-arranges data
or not. In the long run, optimizing these components will involve detecting
such situations where they can avoid performing unnecessary iterations and data
manipulation.
\fi
