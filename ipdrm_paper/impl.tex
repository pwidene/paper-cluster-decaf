\section{Implementation}
\label{s:impl}

Each component is an MPI executable
written in C/C++, varying in length from
191 lines of code (Histogram) to
459 (Select).
Their code is publicly available
in~\cite{champsaur:superglue-repo}.
The processes that make up a single
component belong to the same MPI communicator
once the component is launched.
For each timestep, these processes
communicate to determine how to partition
the overall incoming dataset so that each
process receives an approximately
equal amount of data.

FlexPath allows for the exchange of data
between different communicators living
in different MPI executables, and this
functionality is presented
to the components through the ADIOS interface.
Other implementation paths
are possible here, requiring mainly a common communication
mechanism and a typed payload. For example, the HDF5 Virtual Object
Layer~\cite{folk:2011:hdf5} and Mercury~\cite{Soumagne:2013:mercury} +
Boost serialization could be used.

ADIOS allows each process
involved in the read operation to specify a
bounding box for the multi-dimensional
array portion it will receive.
FlexPath carries out the actual MxN exchange of data,
ensuring each reading process receives
all the data that it specifes in its bounding box,
even if this portion is itself partitioned among several writers.

More generally, FlexPath implements a publish/subscribe,
asynchronous, \textit{stream-based} data exchange abstracted to the
components through the ADIOS interface. This facilitates
the assembly of \sys workflows in a number of ways:

1. Specifying the input and output stream
names as command-line parameters to the \sys components 
when they are launched allows the user
to connect any number of components into
a full workflow.

2. Workflow components can be
launched in any order: downstream components
will wait for data from 
upstream components and upstream
components will buffer data up to a certain
size until they are able to send it
downstream.

3. Even if the number of processes used for one
component is different from that used for the previous
one in the workflow, each component can split the data
(and therefore the computation) evenly among its processes.

4. A FlexPath stream is implemented as
writer side internal data buffering 
until readers are ready to request the data,
at which point a separate writer side thread carries out the data exchange.
This asynchronous model allows a 
\sys component to move on to the computation
of a subsequent timestep when a downstream
component is not yet ready to accept data,
effectively overlapping computation and IO
and offering high performance to a componentized workflow.

There is no need to re-compile \sys components when using them
in different workflows. Any configuration
of a component for a particular workflow
can be provided through run-time arguments.

To enable a scientific code to drive a workflow using our
\sys components, one needs to modify its output routines to use ADIOS.
Specifically, ADIOS expects multi-dimensional
arrays to be packed linearly, with the variables
describing the dimensions
specified in an XML configuration file
that is read by ADIOS at run time.
Roughly 70 lines of code were required
to allow each of the three simulations
in our evaluation, namely LAMMPS, GTCP, and GROMACS,
to work with \sys,
along with an approximately 25-line
XML file.
This small, one-time modification to a simulation allows it to work with any
\sys workflow. Other implementations would have similar requirements.
