\begin{abstract}


Multi-step scientific workflows have become prominent and powerful tools of
data-driven scientific discovery. Run-time analytic techniques are now commonly
used to mitigate the performance effects of using parallel file systems as
staging areas during workflow execution. However, workflow construction and
deployment for extreme-scale computing is still largely an \textit{ad hoc}
process with uneven support from existing tools. In this paper, we present \sys,
an approach to designing generic, reusable components for end-to-end construction
of workflows. Specifically, we demonstrate that a small set of \sys generic components
can be reused to build a diverse set of workflows, using examples based on actual analytic
processes used in three well-known scientific codes. Our evaluation shows
negligible overheads for using a modular approach over a custom, ``all-in-one''
solution.  As extreme-scale systems become
increasingly used for data analysis as well as simulation, tools such as \sys
will become increasingly valuable for defining and deploying flexible, efficient
workflows.


\if 0
Scientific simulation workflows have traditionally used the storage array as a
staging place between the components that make up the workflows and employed a
high-level task manager to trigger these components.
Recent work has been investigating moving
these workflows to purely online models, thus avoiding the well-known
performance bottleneck of storage arrays.  One key advantage of moving to a
purely online model is the ability to launch the entire workflow as part of a
single submission, thus avoiding
the challenges that existing workflow management systems face
in launching individual components on HPC platforms. Still, there lacks
a generic way to assemble complete workflows, even with
many workflows using similar analytical routines.

In this work, we present Superglue, an approach for
assembling entire workflows from existing components
and submitting workflows as single jobs, all through
a single job submission shell script.
We leverage the self-describing, process size- and placement-independent
properties of data exchanged by the Adaptable I/O System (ADIOS) as well as the
MxN publish-subscribe model of Flexpath to assemble and run three Superglue
workflows based on three different and widely-used simulation codes exporting
different data formats.  We show that the performance cost of a componentized
approach to assembling workflows is small, especially when considering the
ability to assemble complete workflows ``out of the box.'' While keeping in
mind performance considerations, we present the building blocks as well as an
approach to expand a library of such components, which we hope can eventually
meet a wide variety of analytical needs in the scientific computing community,
in the common scenario of submitting a workflow as a single job to a
supercomputer scheduler.
\fi

\end{abstract}

\if 0
The workflows are assembled using
some of the same components, but leading to very different analytical results.
In this way, we present the building blocks of a library of generic components


Scientific simulation workflows are becoming a 
Considering the well-known bottleneck that is the persistent storage array,


Scientific simulation workflows have traditionally focused on using the storage
array as a staging place between components and employ a high level task
manaager to trigger components. Recent work has been investigating moving these
workflows to purely online models avoiding the storage array bottlenecks. This
transition is not cost free, but there are also potentially enormous advantages.
For example, by moving online, more the data movement is limited by bisection
bandwidth rather than storage bandwidth. However, making existing tools work
online may take some modification and adding in the ``glue'' components required
to translate the output from one component to the input for the next are no
longer simple scripts.

This paper examines these tradeoffs and lays out when the advantages may make
moving purely online worthwhile even though not being forced by platform
limitations. Support tools necessary to support both models are also discussed
including offering standardized ``glue'' components for online workflows making
the transition easier.
\fi
